\documentclass[]{article}

\begin{document}

\title{Title}
\author{Author}
\date{Today}
\maketitle

\section{Project Activities} 

I will use ants as a model system to illucidate best practices for time-scaling phylogenetic trees from combined molecular and morphological (fossil and neontological) data. For this project, I will use previously published molecular and morphological datasets, and augment these data with new data from the ant collections at the Field Museum. I will also use simulation to generate datasets to examine the effect of various fossil record biases on divergence time estimation. The results of this project will be of interest both to empiricists and to theoreticians looking to improve methods for time-calibrating phylogenetic trees.
The first component of my proposed research will be to build a time-scaled tree of ants. I will accomplish this using new, 'total evidence' methods, which model fossils on a tree as part of a coherent speciation and extinction process along a tree. These newer methods allow for more incomplete fossils to be used in time-scaling, increasing the total number of specimens that can be used in an analysis. Fossil ants, if sufficiently complete, will be added to the matrix of phylogenetic data used to estimate the tree. More incomplete fossils will be used to time-scale the tree.
In the second component of the project, I will use simulations to address the effects of geographical biases on divergence date estimation. Newer, total-evidence models, treat fossils as part of a coherent speciation and extinction process that generates the molecular and morphological data. Recovery of ant fossils is geographically biased. I will use simulations to address the effect of geographical biases in fossil recovery on the estimation of model parameters. 

\section{Intellectual Merit}

Performing total-evidence dating of ants will be a useful comparison to the existing ant phylogenetic trees, which have mostly been based on molecules and a limited set of fossils. Because we can incorporate more ant fossils, including highly-incomplete ants, and ants that are difficult to place within a specific group, we would expect to utilize the fossil resources available more completely. Using more of the data is likely to increase the precision of divergence dating analysis. Ants are a large group, with many successful symbioses, and understanding their evolution is key to understanding the evolution of their interacting partners. This tree will be useful not only to ant biologists, but to agricultural scientists studying the effects of ants on crops, wildlife biologists studying ant-fungal symbiosis and other ecologists and evolutionary biologists.
At the present, using total-evidence methods to estimate phylogenetic trees is a new field. Both components of my project will provide practical reccomendations for biologists and paleontologists for involving fossils in their divergence time estimates.
Particularly, looking at the effect of biases on total evidence methods will provide an avenue for future methods developers, including myself, to improve the performance of these models.
\section{Broader Impacts}

I am an evolutionary biologist who uses computation to study a variety of questions. I am involved in a variety of efforts to increase the computational skills of my fellow scientists. I am an instructor with the Software Carpentry Foundation, and am working with local scientists to offer computational workshops (\url{http://wrightaprilm.github.io/2015-11-16-ISU/}, \url{https://github.com/phyloworks/CyPySci}). I am also involved with the diversity committees of both the Software Carpentry Foundation and the SciPy meeting committee (\url{conference.scipy.org/proceedings/}). Both Dr.\ Heath and Dr.\ Moreau work with women and underrepresented minorities and will provide me with opportunities to offer my workshops for those populations. 

\end{document}